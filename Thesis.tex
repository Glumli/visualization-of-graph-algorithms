\documentclass
[
	paper = a4,
    pagesize,
	12 pt,
	oneside,                       % Chose ›oneside‹ for digital version, ›twoside‹ for printed version.
    open = right,
	DIV = calc,
	BCOR = 0 mm,                   % Binding correction. Only necessary for printed version. Depends on actual binding.
	bibtotoc
]
{scrbook}

% Do not change the following commands.
\newcommand*{\printTitle}{}
\newcommand*{\printAuthor}{}
\newcommand*{\printDateOfBirth}{}
\newcommand*{\printPlaceOfBirth}{}
\newcommand*{\printSubject}{}
\newcommand*{\myTitle}[1]{\renewcommand*{\printTitle}{#1}}
\newcommand*{\myName}[1]{\renewcommand*{\printAuthor}{#1}}
\newcommand*{\myDateOfBirth}[1]{\renewcommand*{\printDateOfBirth}{#1}}
\newcommand*{\myPlaceOfBirth}[1]{\renewcommand*{\printPlaceOfBirth}{#1}}
\newcommand*{\mySubject}[1]{\renewcommand*{\printSubject}{#1}}

\myTitle{High-Level visualization of graph algorrithms}  % Change!
\myName{Julius Milian Severin}  % Change!
\myDateOfBirth{March~12, 1995}  % Change!
\myPlaceOfBirth{Berlin, Germany}  % Change!
\mySubject{{Graph Algorithms}{Visualization}}  % Change!

\usepackage[utf8]{inputenc}
\usepackage[T1]{fontenc}
\usepackage[english]{babel}

\usepackage{graphicx}

\graphicspath{{Images/}}

\usepackage{pgf}

\usepackage{float}


\makeatother

\usepackage							% Microtypography tuning.
[
	protrusion = true,
	expansion = false,
	tracking = true,
	kerning = true,
	spacing = false,
	babel = true
]
{microtype}							% http://ctan.mirrorcatalogs.com/macros/latex/contrib/microtype/microtype.pdf

\SetTracking[unit = space]{font = */*/*/sc/*}{25}   % Adjust kerning for small caps.

\SetExtraKerning[unit = space]		% Adjusted kerning for certain characters.
{
	font = */*/*/*/*
}
{
	: = {100, },
	; = {100, },
	? = {150, 150},
	! = {150, 150},
	: = {250, },
	; = {150, },
	? = {250, 250},
	! = {250, 250},
	» = { , -200},
	« = {-200, },
	› = { , -200},
	‹ = {-200, },
	– = {200, 250},
	— = {200, 250},
	@ = {200, 200}
}

\usepackage{booktabs}
\usepackage{breakurl}
\usepackage{emptypage}

\usepackage[bottom]{footmisc}
\usepackage{remreset}

\makeatletter
    \@removefromreset{footnote}{chapter}
\makeatother

\renewcommand*{\footnoterule}{\rule{0 pt}{0 pt}}
\deffootnote[1.2 em]{1.2 em}{0 em}{\makebox[1.4 em][l]{\textbf{\thefootnotemark}}}

\usepackage{xparse}

\DeclareDocumentCommand{\myfootnote}{o o o m}
{%
    \IfNoValueTF{#1}%
    {%
        \footnote{#4}%
    }%
    {%
        \IfNoValueTF{#2}%
        {%
            \kern #1 em\footnote{#4}%
        }%
        {%
            \IfNoValueTF{#3}%
            {%
                \kern #1 em\footnote{#4}\kern #2 em%
            }%
            {%
                \kern #1 em\footnote[#3]{#4}\kern #2 em%
            }%
        }%
    }%
}

\usepackage{titlesec}

\titleformat{\chapter}
    {\normalfont\rmfamily\huge\bfseries}
    {\thechapter}{1 em}{}

\titleformat{\section}
    {\normalfont\rmfamily\Large\bfseries}
    {\thesection}{1 em}{}

\titleformat{\subsection}
    {\normalfont\rmfamily\large\bfseries}
    {\thesubsection}{1 em}{}

\titleformat{\subsubsection}
    {\normalfont\rmfamily\normalsize\bfseries}
    {\subsubsectionname}{1 em}{}


\pagestyle{headings}

\usepackage{amsmath}
\usepackage{amssymb}
\usepackage{amsfonts}
%\usepackage{upgreek}               % Use if you want to use upright lowercase and italic upercase greek letters.
%\usepackage{dsfont}                % Use if you want to use special symbols.

%\usepackage[printonlyused]{acronym} % Use if you want to have acronyms. http://mirror.hmc.edu/ctan/macros/latex/contrib/acronym/acronym.pdf

%\usepackage{pifont}				% Special symbols.
%\usepackage{fourier-orns}			% More special symbols.
\usepackage{lettrine}

\usepackage{enumitem}
\usepackage
[
	format = plain,
	textfont = {sf, footnotesize},
	labelfont = {sf, bf}
]
{caption}[2008/08/24]
\usepackage{subcaption}				% For using sub-figures.

\usepackage
[
    algo2e,
    ruled,
    vlined,
    linesnumbered,
    algochapter
]
{algorithm2e}

\SetAlCapFnt{\sffamily\footnotesize}
\SetAlCapNameFnt{\sffamily\footnotesize}

\usepackage[numbers]{natbib}

\makeatletter
    \def\NAT@spacechar{~}% NEW
\makeatother

\definecolor{darkblue}{rgb}{0, 0, 0.5}

\makeatletter
\let\latex@xfloat=\@xfloat
\def\@xfloat #1[#2]{%
  \latex@xfloat #1[#2]%
  \def\baselinestretch{1}
  \@normalsize\normalsize
  \normalsize
}

\usepackage{hyperref}
\hypersetup{
	bookmarks = true,
	bookmarksopen = false,
	bookmarksnumbered = true,
	pdfstartpage = 1,
	pdftitle = {{\printTitle}},
	pdfauthor = {{\printAuthor}},
	pdfsubject = {{\printSubject}},
	backref = page,
	breaklinks = true,
	colorlinks = true,
	linkcolor = darkblue,
	anchorcolor = darkblue,
	citecolor = darkblue,
	filecolor = darkblue,
	menucolor = darkblue,
	pagecolor = darkblue,
	urlcolor = darkblue
}


\providecommand*{\backref}[1]{}
\providecommand*{\backrefalt}[4]
{
    \ifcase #1
        Not cited.
    \or
        \footnotesize (Cited on page #2.)
    \else
        \footnotesize (Cited on pages #2.)
    \fi
}

\usepackage{cleveref}
\usepackage{todonotes}




\begin{document}


\frontmatter
% Title Page
\thispagestyle{empty}
\begin{center}
    {\Huge \textbf{\printTitle}}\\[4 ex]
    {\Large \textbf{\printGermanTitle}}\\[4 ex]
    {\Large\textsc{Bachelor Thesis}}\\[4 ex]
    {\large
        to attain the scientific degree\\[4 ex]
        \textbf{Bachelor of Science}
    }
   \\[5 ex]
    \vfill
    \includegraphics[width = 8 cm]{HPI_logo.pdf}\\[8 ex]
    \begin{tabular}{l}
        handed in by \textbf{\printAuthor}\\[1.1 ex]
        born \printDateOfBirth, in \printPlaceOfBirth\\[3 ex]
        \begin{tabular}{@{}ll@{}}
  \textbf{Advisors:} & Prof.~Dr.~Tobias Friedrich\\
              & \printAdvisorOne\\
              & \printAdvisorTwo
        \end{tabular}
        \\[5 ex]
    \end{tabular}

    Chair of Algorithm Engineering\\[5 ex]

    Potsdam, \today
\end{center}
%
% \thispagestyle{empty}
% \begin{center}
%     {\Huge \textbf{\printTitle}}\\[7 ex]
%     {\Large\textsc{Bachelor Thesis}}\\[4 ex]
%     {\large
%         to attain the scientific degree\\[4 ex]
%         \textbf{Bachelor of Science}
%     }
%     \vfill
%     \includegraphics[width = 8 cm]{HPI_logo.pdf}\\[8 ex]
%     \begin{tabular}{l}
%         handed in by \textbf{\printAuthor}\\[1.1 ex]
%         born \printDateOfBirth, in \printPlaceOfBirth\\[3 ex]
%         \textbf{Advisor:} Prof.~Dr.~Tobias Friedrich\\[9 ex]
%     \end{tabular}
%
%     Potsdam, \today
% \end{center}


% Abstract
\chapter*{}
\addcontentsline{toc}{chapter}{Abstract}
\thispagestyle{empty}

\begin{center}
    \large \textbf{Abstract}
\end{center}

Developing an algorithm is an iterative process.
Therefore an understanding of the different approches is important.
Furthermore, as the algorithm becomes more complex, it may come to different results than expected, when running on real data.

This thesis deals with those issues on the example of shortest path algorithm on tiled map data.
Therefore, multiple ways of visualizing a shortest path algorithm with respect of this specific class of algorithms are introduced.
Using those methods we were able to archieve a better understanding of the algorithms and built a powerful basis for discussions.
In addition we enabled the researchers to find the missbehaviour more easily.

\chapter*{}
\addcontentsline{toc}{chapter}{Zusammenfassung}
\thispagestyle{empty}

\begin{center}
    \large \textbf{Zusammenfassung}
\end{center}

Das Entwickeln eines Algorithmusses ist ein iterativer Prozess.
Daher ist es wichitg, dass alle Beteiligten, die verschiedenen Ansätze verstehen.
Ausserdem kann es mit wachsender Komplexität des Algoithmus dazu kommen, dass sich der Algorithmus auf echten Daten anders verhalten als gedacht.
Das finden der Ursache kann ein äußerst komplexes Verfahren sein.

In der folgenden Arbeit erklären wir, wie wir die Entwicklung eines Algorithmus zum finden kürzester Wege durch eine Visualisierung untersützt haben.
Dabei gehen wir auf die Probleme ein, die wir wärend der Entwicklung der Visualisierung lösen mussten.


% Table of Contents
\renewcommand*{\listfigurename}{\normalfont\rmfamily\Large\bfseries List of Figures}
\renewcommand*{\listtablename}{\normalfont\rmfamily\Large\bfseries List of Tables}
\pdfbookmark{\contentsname}{toc}
\addtokomafont{disposition}{\rmfamily}
\microtypesetup{protrusion = false}
    \vspace{-2cm}
    \tableofcontents
    \begingroup
    \let\clearpage\relax
%    \listoffigures                  % Comment out if necessary.
%    \listoftables                   % Comment out if necessary.
    \enlargethispage{1cm}
    \endgroup
\microtypesetup{protrusion = true}



\mainmatter
\chapter{Introduction}


Graphs are common structure in comuter science.
Therefore developing more efficient graph algorithms plays a big role in algorithm engineering.
\newline
For the last two semesters the bachelor project of the algorithms chair, that I was a part of, has been working on a shortest path algortihm for navgiation in road graph.
The project was about developing an algorithm that handles the Navigation Data Standart (NDS) in an efficient way. In NDS the graph is splittet in tiles according to a geographical grid. Those tiles are encrypted and compressed. Therefore the major cost factor during the algorithm is the amount of tiles that has to be loaded in the cache.
\newline
When developing an algorithm getting an idea of how the different approaches behave on real graph data is extremly important.
On the one hand developing an algorithm is an iterative process. Hence it is important to communicate about the different approaches and therefore create an understanding of them in the whole team.
On the other hand the developed algorithms can become quite complex the idea of how the algorithms should behave and the way they actually behave on real graph data can diverge quite alot.
\\
For this we started writing some information about the nodes and tile processed by the algorithms to a text file.
Looking at those was an unefficient and annoying task. Therefore we started developing a visualization.\\
In the following I will describe the process of developing a sencemaking visualization using the examble of our project.


\chapter{Gedanken}
NDS motivieren?
Inwieweit das Projekt erklären?
Einzelne Features:
Wie genau auf Algorithmen eingehen?
Nur verlinken?
Erst Wünsche an Visualisierung?

Historisch rangehen?\\

Introduction\\
Graphs important -> developing graph algorithms important -> Bad results/hard to imagine how the algorithm works in reality -> Visualizeing helps to access information -> Visualizing the algorithm -> bachelorproject -> problem -> building a visualization based on problem\\
Main part\\
historische rangehensweise? --> Motivation --> resultat --> relektieren --> repeat\\
Fazit\\
Resultat verallgemeinern --> Maß zwischen planen und baunen finden --> schnell beginnen --> feedback von nutzern\\

\chapter{Main}

\section{Displaying Edges}
- good only shows the searchfront\\
- slow\\
- no value on big scale\\
\subsection{Showing the watched edges} \label{edges only}

The first visualization that comes in mind is to only display the edges that has been watched already. This leads to an impressing image of the search front and how it develops during the algorithm but multiple questions stay unanswered.\\
When the algorithm watches already explored edges the user doesn't see any movement on the screen.\\
In most cases there is no sence in having every single edge displayed.
Only the outer tiles are easily recognizeable.

\subsection{Showing the watched edges and the displayed blocks}
As mentionioned in \ref{edges only} the tile structure of the map is totaly ignored in the visualization. As the tiles play a major role in the project they should also play a major role in the visualization.
So the next step was to also draw the tiles when they have been loaded.

\section{Displaying Blocks}
- faster\\

\subsection{Time based coloring}

\subsection{Cache based coloring}

\section{Displaying Graph in background}

\section{Diff between Searchfronts}

\section{basic features}
- zoom\\
- stepping threw time\\


\chapter{Chapter}

\lettrine[findent = -0.3 em, nindent = 0.7 em]{A}{} small example as proposed by~\citet{test}.\myfootnote[-0.2]{The footnote number shouldn’t be in superscript down here.} Just some text referring to~\Cref{fig:HPI} and citing~\citet{test2}.\myfootnote[-0.2]{And another footnote.}

\begin{figure}
    \centering
    \includegraphics[width = 9 cm]{HPI_Logo.pdf}
    \caption[HPI logo]{\label{fig:HPI}That’s the logo of the HPI. Naturally, it comes without its obligatory safe zone.}
\end{figure}

\section{Section}

\subsection{Subsection}

\subsubsection{Subsubsection}

\newpage
More text.
\newpage
Even more text.

% References
\renewcommand*{\bibname}{References}
\bibliographystyle{abbrvnat}
\bibliography{Files/References}

\chapter*{Independence Declaration}
\addcontentsline{toc}{chapter}{Declaration of Authorship}
\thispagestyle{empty}

I hereby declare that the thesis submitted is my own unaided work. All direct or indirect sources used are acknowledged as references.\vspace{2 ex}

Potsdam, \today\\[6 ex]

\begin{flushleft}
    \begin{tabular}{p{5cm}}
        \hline
        \centering\footnotesize\printAuthor
    \end{tabular}
\end{flushleft}


\end{document}
