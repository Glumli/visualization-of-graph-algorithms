\chapter*{}
\addcontentsline{toc}{chapter}{Abstract}
\thispagestyle{empty}

\begin{center}
    \large \textbf{Abstract}
\end{center}

Developing an algorithm is an iterative process.
Therefore an understanding of the different approches is important.
Furthermore, as the algorithm becomes more complex, it may come to different results than expected, when running on real data.

This thesis deals with those issues on the example of shortest path algorithm on tiled map data.
Therefore, multiple ways of visualizing a shortest path algorithm with respect of this specific class of algorithms are introduced.
Using those methods we were able to archieve a better understanding of the algorithms and built a powerful basis for discussions.
In addition we enabled the researchers to find the missbehaviour more easily.

\chapter*{}
\addcontentsline{toc}{chapter}{Zusammenfassung}
\thispagestyle{empty}

\begin{center}
    \large \textbf{Zusammenfassung}
\end{center}

Das Entwickeln eines Algorithmusses ist ein iterativer Prozess.
Daher ist es wichitg, dass alle Beteiligten, die verschiedenen Ansätze verstehen.
Ausserdem kann es mit wachsender Komplexität des Algoithmus dazu kommen, dass sich der Algorithmus auf echten Daten anders verhalten als gedacht.
Das finden der Ursache kann ein äußerst komplexes Verfahren sein.

In der folgenden Arbeit erklären wir, wie wir die Entwicklung eines Algorithmus zum finden kürzester Wege durch eine Visualisierung untersützt haben.
Dabei gehen wir auf die Probleme ein, die wir wärend der Entwicklung der Visualisierung lösen mussten.
