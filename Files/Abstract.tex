\chapter*{}
\addcontentsline{toc}{chapter}{Abstract}
\thispagestyle{empty}

\begin{center}
    \large \textbf{Abstract}
\end{center}

Developing an algorithm is a complex, iterative process.
Therefore an understanding of the different approaches is important.
This is a challenge, in particular when working as a team.
Furthermore, as the algorithms become more complex, they might lead to different results than initially expected, when running on real data.

%This thesis deals with those issues on the example of shortest path algorithm on tiled map data.
This thesis deals with those issues; an example is presented based on developing shortest path algorithms on tiled map data.
Therefore, multiple ways of visualizing a shortest path algorithm, with respect to this specific class of algorithms, are introduced.
Using those methods we were able to achieve a better understanding of the algorithms and built a powerful basis for discussions.
%In addition, we enabled the researchers locate misbehaviours and find their the reasons more easily.
In addition, we made it possible for all research participants to identify erratic behaviour and thus find the source of error more easily.

\chapter*{}
\addcontentsline{toc}{chapter}{Zusammenfassung}
\thispagestyle{empty}

\begin{center}
    \large \textbf{Zusammenfassung}
\end{center}

Das Entwickeln eines Algorithmus ist ein komplexer, iterativer Prozess.
Daher ist es wichitg die verschiedenen Ansätze verstehen.
Das ist gerade dann eine Herausvorderung, wenn in einer Gruppe gearbeitet wird.
Ausserdem kann es mit wachsender Komplexität der Algorithmen dazu kommen, dass sich die Algorithmen auf echten Daten anders verhalten als erwartet.

Diese Arbeit beschäftigt sich mit diesen Problemen am Beispiel eines Algorithmus zum finden kürzester Wege auf in Kacheln unterteilten Karten.
Dafür werden verschiedene Verfahren zur Visualizierung von Kürzeste-Wege-Algorithmen vorgestellt.
Dies geschieht unter Berücksichtigung der speziellen Klasse von Algorithmen.
Durch die Verwendung dieser Methoden ist es uns gelungen ein besseres Verständnis der Algorithmen zu erreichen und eine mächtige Basis für Diskussionen zu schaffen.
Ausserdem haben wir den Forschern ermöglicht Fehlverhalten des Algorithmus festzustellen und deren Ursachen zu finden.
